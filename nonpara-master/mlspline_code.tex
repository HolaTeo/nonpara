\PassOptionsToPackage{unicode=true}{hyperref} % options for packages loaded elsewhere
\PassOptionsToPackage{hyphens}{url}
%
\documentclass[
]{article}
\usepackage{lmodern}
\usepackage{amssymb,amsmath}
\usepackage{ifxetex,ifluatex}
\ifnum 0\ifxetex 1\fi\ifluatex 1\fi=0 % if pdftex
  \usepackage[T1]{fontenc}
  \usepackage[utf8]{inputenc}
  \usepackage{textcomp} % provides euro and other symbols
\else % if luatex or xelatex
  \usepackage{unicode-math}
  \defaultfontfeatures{Scale=MatchLowercase}
  \defaultfontfeatures[\rmfamily]{Ligatures=TeX,Scale=1}
\fi
% use upquote if available, for straight quotes in verbatim environments
\IfFileExists{upquote.sty}{\usepackage{upquote}}{}
\IfFileExists{microtype.sty}{% use microtype if available
  \usepackage[]{microtype}
  \UseMicrotypeSet[protrusion]{basicmath} % disable protrusion for tt fonts
}{}
\makeatletter
\@ifundefined{KOMAClassName}{% if non-KOMA class
  \IfFileExists{parskip.sty}{%
    \usepackage{parskip}
  }{% else
    \setlength{\parindent}{0pt}
    \setlength{\parskip}{6pt plus 2pt minus 1pt}}
}{% if KOMA class
  \KOMAoptions{parskip=half}}
\makeatother
\usepackage{xcolor}
\IfFileExists{xurl.sty}{\usepackage{xurl}}{} % add URL line breaks if available
\IfFileExists{bookmark.sty}{\usepackage{bookmark}}{\usepackage{hyperref}}
\hypersetup{
  pdftitle={Hwang's mlspline},
  pdfauthor={Choi TaeYoung},
  pdfborder={0 0 0},
  breaklinks=true}
\urlstyle{same}  % don't use monospace font for urls
\usepackage[margin=1in]{geometry}
\usepackage{color}
\usepackage{fancyvrb}
\newcommand{\VerbBar}{|}
\newcommand{\VERB}{\Verb[commandchars=\\\{\}]}
\DefineVerbatimEnvironment{Highlighting}{Verbatim}{commandchars=\\\{\}}
% Add ',fontsize=\small' for more characters per line
\usepackage{framed}
\definecolor{shadecolor}{RGB}{248,248,248}
\newenvironment{Shaded}{\begin{snugshade}}{\end{snugshade}}
\newcommand{\AlertTok}[1]{\textcolor[rgb]{0.94,0.16,0.16}{#1}}
\newcommand{\AnnotationTok}[1]{\textcolor[rgb]{0.56,0.35,0.01}{\textbf{\textit{#1}}}}
\newcommand{\AttributeTok}[1]{\textcolor[rgb]{0.77,0.63,0.00}{#1}}
\newcommand{\BaseNTok}[1]{\textcolor[rgb]{0.00,0.00,0.81}{#1}}
\newcommand{\BuiltInTok}[1]{#1}
\newcommand{\CharTok}[1]{\textcolor[rgb]{0.31,0.60,0.02}{#1}}
\newcommand{\CommentTok}[1]{\textcolor[rgb]{0.56,0.35,0.01}{\textit{#1}}}
\newcommand{\CommentVarTok}[1]{\textcolor[rgb]{0.56,0.35,0.01}{\textbf{\textit{#1}}}}
\newcommand{\ConstantTok}[1]{\textcolor[rgb]{0.00,0.00,0.00}{#1}}
\newcommand{\ControlFlowTok}[1]{\textcolor[rgb]{0.13,0.29,0.53}{\textbf{#1}}}
\newcommand{\DataTypeTok}[1]{\textcolor[rgb]{0.13,0.29,0.53}{#1}}
\newcommand{\DecValTok}[1]{\textcolor[rgb]{0.00,0.00,0.81}{#1}}
\newcommand{\DocumentationTok}[1]{\textcolor[rgb]{0.56,0.35,0.01}{\textbf{\textit{#1}}}}
\newcommand{\ErrorTok}[1]{\textcolor[rgb]{0.64,0.00,0.00}{\textbf{#1}}}
\newcommand{\ExtensionTok}[1]{#1}
\newcommand{\FloatTok}[1]{\textcolor[rgb]{0.00,0.00,0.81}{#1}}
\newcommand{\FunctionTok}[1]{\textcolor[rgb]{0.00,0.00,0.00}{#1}}
\newcommand{\ImportTok}[1]{#1}
\newcommand{\InformationTok}[1]{\textcolor[rgb]{0.56,0.35,0.01}{\textbf{\textit{#1}}}}
\newcommand{\KeywordTok}[1]{\textcolor[rgb]{0.13,0.29,0.53}{\textbf{#1}}}
\newcommand{\NormalTok}[1]{#1}
\newcommand{\OperatorTok}[1]{\textcolor[rgb]{0.81,0.36,0.00}{\textbf{#1}}}
\newcommand{\OtherTok}[1]{\textcolor[rgb]{0.56,0.35,0.01}{#1}}
\newcommand{\PreprocessorTok}[1]{\textcolor[rgb]{0.56,0.35,0.01}{\textit{#1}}}
\newcommand{\RegionMarkerTok}[1]{#1}
\newcommand{\SpecialCharTok}[1]{\textcolor[rgb]{0.00,0.00,0.00}{#1}}
\newcommand{\SpecialStringTok}[1]{\textcolor[rgb]{0.31,0.60,0.02}{#1}}
\newcommand{\StringTok}[1]{\textcolor[rgb]{0.31,0.60,0.02}{#1}}
\newcommand{\VariableTok}[1]{\textcolor[rgb]{0.00,0.00,0.00}{#1}}
\newcommand{\VerbatimStringTok}[1]{\textcolor[rgb]{0.31,0.60,0.02}{#1}}
\newcommand{\WarningTok}[1]{\textcolor[rgb]{0.56,0.35,0.01}{\textbf{\textit{#1}}}}
\usepackage{graphicx,grffile}
\makeatletter
\def\maxwidth{\ifdim\Gin@nat@width>\linewidth\linewidth\else\Gin@nat@width\fi}
\def\maxheight{\ifdim\Gin@nat@height>\textheight\textheight\else\Gin@nat@height\fi}
\makeatother
% Scale images if necessary, so that they will not overflow the page
% margins by default, and it is still possible to overwrite the defaults
% using explicit options in \includegraphics[width, height, ...]{}
\setkeys{Gin}{width=\maxwidth,height=\maxheight,keepaspectratio}
\setlength{\emergencystretch}{3em}  % prevent overfull lines
\providecommand{\tightlist}{%
  \setlength{\itemsep}{0pt}\setlength{\parskip}{0pt}}
\setcounter{secnumdepth}{-2}
% Redefines (sub)paragraphs to behave more like sections
\ifx\paragraph\undefined\else
  \let\oldparagraph\paragraph
  \renewcommand{\paragraph}[1]{\oldparagraph{#1}\mbox{}}
\fi
\ifx\subparagraph\undefined\else
  \let\oldsubparagraph\subparagraph
  \renewcommand{\subparagraph}[1]{\oldsubparagraph{#1}\mbox{}}
\fi

% set default figure placement to htbp
\makeatletter
\def\fps@figure{htbp}
\makeatother

% https://github.com/rstudio/rmarkdown/issues/337
\let\rmarkdownfootnote\footnote%
\def\footnote{\protect\rmarkdownfootnote}

% https://github.com/rstudio/rmarkdown/pull/252
\usepackage{titling}
\setlength{\droptitle}{-2em}

\pretitle{\vspace{\droptitle}\centering\huge}
\posttitle{\par}

\preauthor{\centering\large\emph}
\postauthor{\par}

\predate{\centering\large\emph}
\postdate{\par}

\title{Hwang's mlspline}
\author{Choi TaeYoung}
\date{2019 11 30}

\begin{document}
\maketitle

\#' generate simulated response for multilevel splines \#' \#' Generates
simulated response for multilevel splines \#' \#' @importFrom stats coef
glm lm rbinom rnorm vcov \#' @param J number of `data' intervals \#'
@param mod underlying model; either \texttt{lm} or \texttt{glm} \#'
@param x\_sigma design matrix sigma \#' @param e\_sigma error variance -
around the mean function; data level. \#' @param z\_sigma error variance
around my surface; structural level. \#' @param N\_s the minimum sample
size for each interval. \#' @param N\_m the maximum sample size for each
interval; default = 200. \#' @return returns a list described above. \#'
@format list(x\_list = x\_list, y\_list = y\_list, e\_list = e\_list,
true\_mu = mu, z = z) \#' \textbackslash{}describe\{ \#'

\item

\{x\_list\}\{the length-J list of design matrices. The nrow of each
element is between N\_s and N\_m\} \#'

\item

\{y\_list\}\{the length-J list of response vectors. The length of each
element is between N\_s and N\_m.\} \#'

\item

\{e\_list\}\{the length-J list of error vectors. The length of each
element is between N\_s and N\_m.\} \#'

\item

\{true\_mu\}\{the true mu vector of length J\} \#'

\item

\{z\}\{the grid vector of length J\} \#'

\begin{Shaded}
\begin{Highlighting}[]
\NormalTok{generate_response <-}\StringTok{ }\ControlFlowTok{function}\NormalTok{(J, mod, }\DataTypeTok{e_sigma =} \DecValTok{1}\NormalTok{, }\DataTypeTok{x_sigma =} \DecValTok{1}\NormalTok{, }\DataTypeTok{z_sigma =} \FloatTok{0.5}\NormalTok{, N_s, }\DataTypeTok{N_m =} \DecValTok{200}\NormalTok{) \{}

  \CommentTok{# currently the data interval (z interval) is set to be between -3 and 3.}

\NormalTok{  n <-}\StringTok{ }\KeywordTok{sample}\NormalTok{(N_s}\OperatorTok{:}\NormalTok{N_m, J, }\DataTypeTok{replace =} \OtherTok{TRUE}\NormalTok{)}

  \CommentTok{# smooth surface: z is the grid sequence and mu is the generated smooth function.}
\NormalTok{  z <-}\StringTok{ }\KeywordTok{seq}\NormalTok{(}\DataTypeTok{from =} \DecValTok{-3}\NormalTok{, }\DataTypeTok{to =} \DecValTok{3}\NormalTok{, }\DataTypeTok{length.out =}\NormalTok{ J)}
\NormalTok{  mu <-}\StringTok{ }\NormalTok{z}\OperatorTok{^}\DecValTok{2} \OperatorTok{-}\StringTok{ }\DecValTok{10} \OperatorTok{*}\StringTok{ }\KeywordTok{cos}\NormalTok{(}\DecValTok{2} \OperatorTok{*}\StringTok{ }\NormalTok{pi }\OperatorTok{*}\StringTok{ }\NormalTok{z)  }\CommentTok{# "true" surface.}

\NormalTok{  beta_}\DecValTok{1}\NormalTok{ <-}\StringTok{ }\NormalTok{mu }\OperatorTok{+}\StringTok{ }\KeywordTok{rnorm}\NormalTok{(J, }\DecValTok{0}\NormalTok{, z_sigma)  }\CommentTok{# slope}
\NormalTok{  beta_}\DecValTok{0}\NormalTok{ <-}\StringTok{ }\DecValTok{0}  \CommentTok{# intercept}

\NormalTok{  x_list <-}\StringTok{ }\KeywordTok{lapply}\NormalTok{(n, rnorm, }\DataTypeTok{mean =} \DecValTok{0}\NormalTok{, }\DataTypeTok{sd =}\NormalTok{ x_sigma)}
\NormalTok{  e_list <-}\StringTok{ }\KeywordTok{lapply}\NormalTok{(n, rnorm, }\DataTypeTok{mean =} \DecValTok{0}\NormalTok{, }\DataTypeTok{sd =}\NormalTok{ e_sigma)}

  \CommentTok{# outcome generation function; gives 'y' list given e, beta_0, beta_1, and}
  \CommentTok{# x (design matrix)}
  \CommentTok{# for glm: logit link binary p(y = 1) = 1/(1 + exp(-beta_0 - beta_1 * x - e)}
  \CommentTok{# for lm: ordinary linear model structure y = xb + e}
  \ControlFlowTok{if}\NormalTok{ (mod }\OperatorTok{==}\StringTok{ "glm"}\NormalTok{) \{}
\NormalTok{    y_list <-}\StringTok{ }\KeywordTok{mapply}\NormalTok{(}\ControlFlowTok{function}\NormalTok{(x, e, b, }\DataTypeTok{beta_0 =} \DecValTok{0}\NormalTok{)}
      \KeywordTok{rbinom}\NormalTok{(}\KeywordTok{length}\NormalTok{(x), }\DecValTok{1}\NormalTok{, }\DecValTok{1}\OperatorTok{/}\NormalTok{(}\DecValTok{1} \OperatorTok{+}\StringTok{ }\KeywordTok{exp}\NormalTok{(}\OperatorTok{-}\NormalTok{beta_}\DecValTok{0} \OperatorTok{-}\StringTok{ }\NormalTok{b }\OperatorTok{*}\StringTok{ }\NormalTok{x }\OperatorTok{-}\StringTok{ }\NormalTok{e))),}
      \DataTypeTok{x =}\NormalTok{ x_list, }\DataTypeTok{e =}\NormalTok{ e_list, }\DataTypeTok{b =}\NormalTok{ beta_}\DecValTok{1}\NormalTok{)}
\NormalTok{  \}}
  \ControlFlowTok{if}\NormalTok{ (mod }\OperatorTok{==}\StringTok{ "lm"}\NormalTok{) \{}
\NormalTok{    y_list <-}\StringTok{ }\KeywordTok{mapply}\NormalTok{(}\ControlFlowTok{function}\NormalTok{(x, e, b, }\DataTypeTok{beta_0 =} \DecValTok{0}\NormalTok{)}
\NormalTok{      beta_}\DecValTok{0} \OperatorTok{+}\StringTok{ }\NormalTok{b }\OperatorTok{*}\StringTok{ }\NormalTok{x }\OperatorTok{+}\StringTok{ }\NormalTok{e, }\DataTypeTok{x =}\NormalTok{ x_list, }\DataTypeTok{e =}\NormalTok{ e_list, }\DataTypeTok{b =}\NormalTok{ beta_}\DecValTok{1}\NormalTok{)}
\NormalTok{  \}}
  \KeywordTok{list}\NormalTok{(}\DataTypeTok{x_list =}\NormalTok{ x_list, }\DataTypeTok{y_list =}\NormalTok{ y_list, }\DataTypeTok{e_list =}\NormalTok{ e_list, }\DataTypeTok{true_mu =}\NormalTok{ mu, }\DataTypeTok{z =}\NormalTok{ z)}
\NormalTok{\}}
\end{Highlighting}
\end{Shaded}

\#' Builds
`\texttt{granular\textquotesingle{}\textquotesingle{}\ data\ \#\textquotesingle{}\ \#\textquotesingle{}\ obtains\ the\ regression\ slope\ and\ its\ variance\ \#\textquotesingle{}\ certainly\ not\ optimal\ but\ this\ step\ shouldn\textquotesingle{}t\ take\ long\ regardless\ \#\textquotesingle{}\ @param\ x\_k\ design\ matrix\ \#\textquotesingle{}\ @param\ y\_k\ response\ vector\ \#\textquotesingle{}\ @param\ mod\ underlying\ model;\ either}lm\texttt{or}glm`
\#' @export

\begin{Shaded}
\begin{Highlighting}[]
\NormalTok{granular <-}\StringTok{ }\ControlFlowTok{function}\NormalTok{(x_k, y_k, mod) \{}
  \CommentTok{# summarizing the regression part}
  \ControlFlowTok{if}\NormalTok{ (mod }\OperatorTok{==}\StringTok{ "glm"}\NormalTok{)}
\NormalTok{    fit_lm <-}\StringTok{ }\KeywordTok{glm}\NormalTok{(y_k }\OperatorTok{~}\StringTok{ }\NormalTok{x_k, }\DataTypeTok{family =} \StringTok{"binomial"}\NormalTok{)}
  \ControlFlowTok{if}\NormalTok{ (mod }\OperatorTok{==}\StringTok{ "lm"}\NormalTok{)}
\NormalTok{    fit_lm <-}\StringTok{ }\KeywordTok{lm}\NormalTok{(y_k }\OperatorTok{~}\StringTok{ }\NormalTok{x_k)}

\NormalTok{  kth_beta_hat <-}\StringTok{ }\KeywordTok{coef}\NormalTok{(fit_lm)[}\DecValTok{2}\NormalTok{]}
\NormalTok{  kth_var <-}\StringTok{ }\KeywordTok{diag}\NormalTok{(}\KeywordTok{vcov}\NormalTok{(fit_lm))[}\DecValTok{2}\NormalTok{]}
\NormalTok{  grain_out <-}\StringTok{ }\KeywordTok{list}\NormalTok{(kth_beta_hat, kth_var)}
\NormalTok{  grain_out}
\NormalTok{\}}
\end{Highlighting}
\end{Shaded}

\#' Generates kerel matrix \#' \#' Generates kernel matrix of J by J,
where J = length(z) for multilevel splines \#' certainly not optimal but
this step shouldn't take long regardless. \#' Used the formulation from
Reinsch (1967). \#' @param z Mid-interval value vector, it is safe to
assume this to be equi-distant, but in principle it doesn't have to be.
it's not tested though. \#' @export

\begin{Shaded}
\begin{Highlighting}[]
\NormalTok{make_K <-}\StringTok{ }\ControlFlowTok{function}\NormalTok{(z) \{}
\NormalTok{  J <-}\StringTok{ }\KeywordTok{length}\NormalTok{(z)}
\NormalTok{  Del <-}\StringTok{ }\KeywordTok{matrix}\NormalTok{(}\DecValTok{0}\NormalTok{, }\DataTypeTok{nrow =}\NormalTok{ J }\OperatorTok{-}\StringTok{ }\DecValTok{2}\NormalTok{, }\DataTypeTok{ncol =}\NormalTok{ J)}
\NormalTok{  W <-}\StringTok{ }\KeywordTok{matrix}\NormalTok{(}\DecValTok{0}\NormalTok{, }\DataTypeTok{nrow =}\NormalTok{ J }\OperatorTok{-}\StringTok{ }\DecValTok{2}\NormalTok{, }\DataTypeTok{ncol =}\NormalTok{ J }\OperatorTok{-}\StringTok{ }\DecValTok{2}\NormalTok{)}
\NormalTok{  h <-}\StringTok{ }\KeywordTok{diff}\NormalTok{(z)}
  \ControlFlowTok{for}\NormalTok{ (l }\ControlFlowTok{in} \DecValTok{1}\OperatorTok{:}\NormalTok{(J }\OperatorTok{-}\StringTok{ }\DecValTok{2}\NormalTok{)) \{}
\NormalTok{    Del[l, l] <-}\StringTok{ }\DecValTok{1}\OperatorTok{/}\NormalTok{h[l]}
\NormalTok{    Del[l, (l }\OperatorTok{+}\StringTok{ }\DecValTok{1}\NormalTok{)] <-}\StringTok{ }\DecValTok{-1}\OperatorTok{/}\NormalTok{h[l] }\OperatorTok{-}\StringTok{ }\DecValTok{1}\OperatorTok{/}\NormalTok{h[(l }\OperatorTok{+}\StringTok{ }\DecValTok{1}\NormalTok{)]}
\NormalTok{    Del[l, (l }\OperatorTok{+}\StringTok{ }\DecValTok{2}\NormalTok{)] <-}\StringTok{ }\DecValTok{1}\OperatorTok{/}\NormalTok{h[(l }\OperatorTok{+}\StringTok{ }\DecValTok{1}\NormalTok{)]}
\NormalTok{    W[(l }\OperatorTok{-}\StringTok{ }\DecValTok{1}\NormalTok{), l] <-}\StringTok{ }\NormalTok{W[l, (l }\OperatorTok{-}\StringTok{ }\DecValTok{1}\NormalTok{)] <-}\StringTok{ }\NormalTok{h[l]}\OperatorTok{/}
\StringTok{      }\DecValTok{6}
\NormalTok{    W[l, l] <-}\StringTok{ }\NormalTok{(h[l] }\OperatorTok{+}\StringTok{ }\NormalTok{h[l }\OperatorTok{+}\StringTok{ }\DecValTok{1}\NormalTok{])}\OperatorTok{/}\DecValTok{3}
\NormalTok{  \}}
\NormalTok{  K <-}\StringTok{ }\KeywordTok{t}\NormalTok{(Del) }\OperatorTok\StringTok{ }\KeywordTok{solve}\NormalTok{(W) }\OperatorTok\StringTok{ }\NormalTok{Del}
\NormalTok{  K}
\NormalTok{\}}
\end{Highlighting}
\end{Shaded}

\#' Main EM function \#' \#' Running EM for multilevel splines \#'
certainly not optimal\ldots{} \#' @param beta\_hat\_vec data vector of
length J \#' @param V covariance matrix of size J by J \#' @param K
kernel matrix from \texttt{make\_K} \#' @param lambda tuning parameter
\#' @param maxit maximum iteration number \#' @export

\begin{Shaded}
\begin{Highlighting}[]
\NormalTok{main_EM <-}\StringTok{ }\ControlFlowTok{function}\NormalTok{(beta_hat_vec, V, K, lambda, }\DataTypeTok{maxit =} \DecValTok{500}\NormalTok{) \{}

  \CommentTok{# parameter initilization}
\NormalTok{  eps <-}\StringTok{ }\DecValTok{1000}  \CommentTok{# convergence tracker}
\NormalTok{  tol <-}\StringTok{ }\FloatTok{1e-05}  \CommentTok{# convergence threshold}
\NormalTok{  sigma2_m <-}\StringTok{ }\KeywordTok{mean}\NormalTok{(}\KeywordTok{diag}\NormalTok{(V))}
\NormalTok{  J <-}\StringTok{ }\KeywordTok{length}\NormalTok{(beta_hat_vec)}
\NormalTok{  mu_m <-}\StringTok{ }\KeywordTok{rep}\NormalTok{(}\KeywordTok{mean}\NormalTok{(beta_hat_vec), J)}
\NormalTok{  I <-}\StringTok{ }\KeywordTok{diag}\NormalTok{(J)}
\NormalTok{  iter <-}\StringTok{ }\DecValTok{1}

  \ControlFlowTok{while}\NormalTok{ (eps }\OperatorTok{>}\StringTok{ }\NormalTok{tol }\OperatorTok{&}\StringTok{ }\NormalTok{iter }\OperatorTok{<=}\StringTok{ }\NormalTok{maxit) \{}
    \CommentTok{# .. EM starts here}
\NormalTok{    mu_m_old <-}\StringTok{ }\NormalTok{mu_m}
\NormalTok{    sigma2_m_old <-}\StringTok{ }\NormalTok{sigma2_m  }\CommentTok{# current sigma^2}

\NormalTok{    Vst <-}\StringTok{ }\KeywordTok{solve}\NormalTok{(}\KeywordTok{solve}\NormalTok{(V) }\OperatorTok{+}\StringTok{ }\NormalTok{(}\DecValTok{1}\OperatorTok{/}\NormalTok{sigma2_m) }\OperatorTok{*}\StringTok{ }\KeywordTok{diag}\NormalTok{(J))  }\CommentTok{# Vst}
\NormalTok{    D_m <-}\StringTok{ }\NormalTok{Vst }\OperatorTok\StringTok{ }\KeywordTok{solve}\NormalTok{(V)  }\CommentTok{#D_m <- part_cov %*% V}
\NormalTok{    mu_m <-}\StringTok{ }\KeywordTok{solve}\NormalTok{(D_m }\OperatorTok{+}\StringTok{ }\NormalTok{lambda }\OperatorTok{*}\StringTok{ }\NormalTok{K) }\OperatorTok\StringTok{ }\NormalTok{D_m }\OperatorTok\StringTok{ }\NormalTok{beta_hat_vec}

\NormalTok{    S_lambda <-}\StringTok{ }\KeywordTok{solve}\NormalTok{(I }\OperatorTok\StringTok{ }\NormalTok{D_m }\OperatorTok\StringTok{ }\NormalTok{I }\OperatorTok{+}\StringTok{ }\NormalTok{lambda }\OperatorTok{*}\StringTok{ }\NormalTok{K) }\OperatorTok\StringTok{ }\NormalTok{I }\OperatorTok\StringTok{ }\NormalTok{D_m}
\NormalTok{    effective_df <-}\StringTok{ }\KeywordTok{sum}\NormalTok{(}\KeywordTok{diag}\NormalTok{(S_lambda))}

\NormalTok{    sigma2_m <-}\StringTok{ }\KeywordTok{mean}\NormalTok{((beta_hat_vec }\OperatorTok{-}\StringTok{ }\NormalTok{mu_m)}\OperatorTok{^}\DecValTok{2}\NormalTok{)}
\NormalTok{    eps <-}\StringTok{ }\KeywordTok{sum}\NormalTok{(}\KeywordTok{abs}\NormalTok{(mu_m }\OperatorTok{-}\StringTok{ }\NormalTok{mu_m_old)) }\OperatorTok{+}\StringTok{ }\KeywordTok{abs}\NormalTok{(sigma2_m_old }\OperatorTok{-}\StringTok{ }\NormalTok{sigma2_m)}
\NormalTok{    iter <-}\StringTok{ }\NormalTok{iter }\OperatorTok{+}\StringTok{ }\DecValTok{1}
    \ControlFlowTok{if}\NormalTok{ (iter }\OperatorTok{==}\StringTok{ }\NormalTok{maxit) \{}
      \KeywordTok{cat}\NormalTok{(}\StringTok{"for lambda ="}\NormalTok{, lambda, }\StringTok{"max iteration reached; may need to double check }\CharTok{\textbackslash{}n}\StringTok{"}\NormalTok{)}
\NormalTok{    \}}
\NormalTok{  \}  }\CommentTok{# end of EM .. convergence reached.}

\NormalTok{  BIC <-}\StringTok{ }\KeywordTok{sum}\NormalTok{((beta_hat_vec }\OperatorTok{-}\StringTok{ }\NormalTok{mu_m)}\OperatorTok{^}\DecValTok{2}\NormalTok{)}\OperatorTok{/}\NormalTok{(J}\OperatorTok{^}\NormalTok{(}\DecValTok{1} \OperatorTok{-}\StringTok{ }\NormalTok{effective_df}\OperatorTok{/}\NormalTok{J))}
\NormalTok{  GCV <-}\StringTok{ }\KeywordTok{sum}\NormalTok{((beta_hat_vec }\OperatorTok{-}\StringTok{ }\NormalTok{mu_m)}\OperatorTok{^}\DecValTok{2}\NormalTok{)}\OperatorTok{/}\NormalTok{(J }\OperatorTok{-}\StringTok{ }\NormalTok{effective_df)}\OperatorTok{^}\DecValTok{2} \OperatorTok{*}\StringTok{ }\NormalTok{J}

\NormalTok{  EM_out <-}\StringTok{ }\KeywordTok{list}\NormalTok{(}\DataTypeTok{mu =}\NormalTok{ mu_m, }\DataTypeTok{S_lambda =}\NormalTok{ S_lambda, }\DataTypeTok{sigma2 =}\NormalTok{ sigma2_m, }\DataTypeTok{BIC =}\NormalTok{ BIC, }\DataTypeTok{GCV =}\NormalTok{ GCV)}
\NormalTok{  EM_out}
\NormalTok{\}}
\end{Highlighting}
\end{Shaded}

\#' Naive strawman \#' \#' Running naive splines \#' @param
beta\_hat\_vec data vector of length J \#' @param K kernel matrix from
\texttt{make\_K} \#' @param lambda tuning parameter \#' @export

\begin{Shaded}
\begin{Highlighting}[]
\NormalTok{naive_ss <-}\StringTok{ }\ControlFlowTok{function}\NormalTok{(beta_hat_vec, lambda, K) \{}

\NormalTok{  J <-}\StringTok{ }\KeywordTok{length}\NormalTok{(beta_hat_vec)}
\NormalTok{  I <-}\StringTok{ }\KeywordTok{diag}\NormalTok{(J)}
\NormalTok{  S_lambda <-}\StringTok{ }\KeywordTok{solve}\NormalTok{(I }\OperatorTok{+}\StringTok{ }\NormalTok{lambda }\OperatorTok{*}\StringTok{ }\NormalTok{K)}
\NormalTok{  f_hat <-}\StringTok{ }\NormalTok{S_lambda }\OperatorTok\StringTok{ }\NormalTok{beta_hat_vec}

\NormalTok{  eff_df <-}\StringTok{ }\KeywordTok{sum}\NormalTok{(}\KeywordTok{diag}\NormalTok{(S_lambda))}

\NormalTok{  GCV <-}\StringTok{ }\KeywordTok{sum}\NormalTok{((beta_hat_vec }\OperatorTok{-}\StringTok{ }\NormalTok{f_hat)}\OperatorTok{^}\DecValTok{2}\NormalTok{)}\OperatorTok{/}\NormalTok{(J }\OperatorTok{-}\StringTok{ }\NormalTok{eff_df)}\OperatorTok{^}\DecValTok{2} \OperatorTok{*}\StringTok{ }\NormalTok{J}
\NormalTok{  BIC <-}\StringTok{ }\KeywordTok{log}\NormalTok{(}\KeywordTok{mean}\NormalTok{((beta_hat_vec }\OperatorTok{-}\StringTok{ }\NormalTok{f_hat)}\OperatorTok{^}\DecValTok{2}\NormalTok{)) }\OperatorTok{+}\StringTok{ }\NormalTok{eff_df }\OperatorTok{*}\StringTok{ }\KeywordTok{log}\NormalTok{(J)}\OperatorTok{/}\NormalTok{J}

\NormalTok{  out <-}\StringTok{ }\KeywordTok{list}\NormalTok{(}\DataTypeTok{mu =}\NormalTok{ f_hat, }\DataTypeTok{S_lambda =}\NormalTok{ S_lambda, }\DataTypeTok{BIC =}\NormalTok{ BIC, }\DataTypeTok{GCV =}\NormalTok{ GCV)}
\NormalTok{  out}
\NormalTok{\}}
\end{Highlighting}
\end{Shaded}

\#' Generates simulated response for multilevel splines -- test function
\#2 \#' \#' @importFrom stats coef glm lm rbinom rnorm vcov \#' @param J
number of `data' intervals \#' @param mod underlying model; either
\texttt{lm} or \texttt{glm} \#' @param x\_sigma design matrix sigma \#'
@param e\_sigma error variance - around the mean function; data level.
\#' @param z\_sigma error variance around my surface; structural level.
\#' @param N\_s the minimum sample size for each interval. \#' @param
N\_m the maximum sample size for each interval; default = 200. \#'
@return returns a list described above. \#' @format list(x\_list =
x\_list, y\_list = y\_list, e\_list = e\_list, true\_mu = mu, z = z) \#'
\describe{
#' This function is supposed to be combined with the other generation function.. but later.
#'   \item{x_list}{the length-J list of design matrices. The nrow of each element is between N_s and N_m}
#'   \item{y_list}{the length-J list of response vectors. The length of each element is between N_s and N_m.}
#'   \item{e_list}{the length-J list of error vectors. The length of each element is between N_s and N_m.}
#'   \item{true_mu}{the true mu vector of length J}
#'   \item{z}{the grid vector of length J}
#' } \#' @export

\begin{Shaded}
\begin{Highlighting}[]
\NormalTok{generate_response_smooth <-}\StringTok{ }\ControlFlowTok{function}\NormalTok{(J, mod, }\DataTypeTok{e_sigma =} \DecValTok{1}\NormalTok{, }\DataTypeTok{x_sigma =} \DecValTok{1}\NormalTok{, }\DataTypeTok{z_sigma =} \FloatTok{0.5}\NormalTok{, N_s, }\DataTypeTok{N_m =} \DecValTok{200}\NormalTok{) \{}

  \CommentTok{# currently the data interval (z interval) is set to be between 0 and 1}

\NormalTok{  n <-}\StringTok{ }\KeywordTok{sample}\NormalTok{(N_s}\OperatorTok{:}\NormalTok{N_m, J, }\DataTypeTok{replace =} \OtherTok{TRUE}\NormalTok{)}

  \CommentTok{# smooth surface: z is the grid sequence and mu is the generated smooth function.}
\NormalTok{  z <-}\StringTok{ }\KeywordTok{seq}\NormalTok{(}\DataTypeTok{from =} \DecValTok{0}\NormalTok{, }\DataTypeTok{to =} \DecValTok{1}\NormalTok{, }\DataTypeTok{length.out =}\NormalTok{ J)}
\NormalTok{  mu <-}\StringTok{ }\KeywordTok{sin}\NormalTok{(}\DecValTok{12}\OperatorTok{*}\NormalTok{(z }\OperatorTok{+}\StringTok{ }\FloatTok{0.2}\NormalTok{)) }\OperatorTok{/}\StringTok{ }\NormalTok{(z }\OperatorTok{+}\StringTok{ }\FloatTok{0.2}\NormalTok{)  }\CommentTok{# "true" surface.}

\NormalTok{  beta_}\DecValTok{1}\NormalTok{ <-}\StringTok{ }\NormalTok{mu }\OperatorTok{+}\StringTok{ }\KeywordTok{rnorm}\NormalTok{(J, }\DecValTok{0}\NormalTok{, z_sigma)  }\CommentTok{# slope}
\NormalTok{  beta_}\DecValTok{0}\NormalTok{ <-}\StringTok{ }\DecValTok{0}  \CommentTok{# intercept}

\NormalTok{  x_list <-}\StringTok{ }\KeywordTok{lapply}\NormalTok{(n, rnorm, }\DataTypeTok{mean =} \DecValTok{0}\NormalTok{, }\DataTypeTok{sd =}\NormalTok{ x_sigma)}
\NormalTok{  e_list <-}\StringTok{ }\KeywordTok{lapply}\NormalTok{(n, rnorm, }\DataTypeTok{mean =} \DecValTok{0}\NormalTok{, }\DataTypeTok{sd =}\NormalTok{ e_sigma)}

  \CommentTok{# outcome generation function; gives 'y' list given e, beta_0, beta_1, and}
  \CommentTok{# x (design matrix)}
  \CommentTok{# for glm: logit link binary p(y = 1) = 1/(1 + exp(-beta_0 - beta_1 * x - e)}
  \CommentTok{# for lm: ordinary linear model structure y = xb + e}
  \ControlFlowTok{if}\NormalTok{ (mod }\OperatorTok{==}\StringTok{ "glm"}\NormalTok{) \{}
\NormalTok{    y_list <-}\StringTok{ }\KeywordTok{mapply}\NormalTok{(}\ControlFlowTok{function}\NormalTok{(x, e, b, }\DataTypeTok{beta_0 =} \DecValTok{0}\NormalTok{)}
      \KeywordTok{rbinom}\NormalTok{(}\KeywordTok{length}\NormalTok{(x), }\DecValTok{1}\NormalTok{, }\DecValTok{1}\OperatorTok{/}\NormalTok{(}\DecValTok{1} \OperatorTok{+}\StringTok{ }\KeywordTok{exp}\NormalTok{(}\OperatorTok{-}\NormalTok{beta_}\DecValTok{0} \OperatorTok{-}\StringTok{ }\NormalTok{b }\OperatorTok{*}\StringTok{ }\NormalTok{x }\OperatorTok{-}\StringTok{ }\NormalTok{e))),}
      \DataTypeTok{x =}\NormalTok{ x_list, }\DataTypeTok{e =}\NormalTok{ e_list, }\DataTypeTok{b =}\NormalTok{ beta_}\DecValTok{1}\NormalTok{)}
\NormalTok{  \}}
  \ControlFlowTok{if}\NormalTok{ (mod }\OperatorTok{==}\StringTok{ "lm"}\NormalTok{) \{}
\NormalTok{    y_list <-}\StringTok{ }\KeywordTok{mapply}\NormalTok{(}\ControlFlowTok{function}\NormalTok{(x, e, b, }\DataTypeTok{beta_0 =} \DecValTok{0}\NormalTok{)}
\NormalTok{      beta_}\DecValTok{0} \OperatorTok{+}\StringTok{ }\NormalTok{b }\OperatorTok{*}\StringTok{ }\NormalTok{x }\OperatorTok{+}\StringTok{ }\NormalTok{e, }\DataTypeTok{x =}\NormalTok{ x_list, }\DataTypeTok{e =}\NormalTok{ e_list, }\DataTypeTok{b =}\NormalTok{ beta_}\DecValTok{1}\NormalTok{)}
\NormalTok{  \}}
  \KeywordTok{list}\NormalTok{(}\DataTypeTok{x_list =}\NormalTok{ x_list, }\DataTypeTok{y_list =}\NormalTok{ y_list, }\DataTypeTok{e_list =}\NormalTok{ e_list, }\DataTypeTok{true_mu =}\NormalTok{ mu, }\DataTypeTok{z =}\NormalTok{ z)}
\NormalTok{\}}
\end{Highlighting}
\end{Shaded}

\hypertarget{simulation-model}{%
\subsection{simulation model}\label{simulation-model}}

\begin{Shaded}
\begin{Highlighting}[]
\NormalTok{ag2 <-}\StringTok{ }\KeywordTok{generate_response_smooth}\NormalTok{(}\DataTypeTok{J=}\DecValTok{50}\NormalTok{, }\DataTypeTok{mod=}\StringTok{"glm"}\NormalTok{, }\DataTypeTok{e_sigma=}\DecValTok{2}\NormalTok{, }\DataTypeTok{N_s=}\DecValTok{50}\NormalTok{)}
\NormalTok{ag22 <-}\StringTok{ }\KeywordTok{generate_response_smooth}\NormalTok{(}\DataTypeTok{J=}\DecValTok{50}\NormalTok{, }\DataTypeTok{mod=}\StringTok{"glm"}\NormalTok{, }\DataTypeTok{e_sigma=}\DecValTok{2}\NormalTok{, }\DataTypeTok{N_s=}\DecValTok{100}\NormalTok{)}
\NormalTok{ag4 <-}\StringTok{ }\KeywordTok{generate_response_smooth}\NormalTok{(}\DataTypeTok{J=}\DecValTok{50}\NormalTok{, }\DataTypeTok{mod=}\StringTok{"glm"}\NormalTok{, }\DataTypeTok{e_sigma=}\DecValTok{4}\NormalTok{, }\DataTypeTok{N_s=}\DecValTok{50}\NormalTok{)}
\NormalTok{ag44 <-}\StringTok{ }\KeywordTok{generate_response_smooth}\NormalTok{(}\DataTypeTok{J=}\DecValTok{50}\NormalTok{, }\DataTypeTok{mod=}\StringTok{"glm"}\NormalTok{, }\DataTypeTok{e_sigma=}\DecValTok{4}\NormalTok{, }\DataTypeTok{N_s=}\DecValTok{100}\NormalTok{)}
\NormalTok{ag8 <-}\StringTok{ }\KeywordTok{generate_response_smooth}\NormalTok{(}\DataTypeTok{J=}\DecValTok{50}\NormalTok{, }\DataTypeTok{mod=}\StringTok{"glm"}\NormalTok{, }\DataTypeTok{e_sigma=}\DecValTok{8}\NormalTok{, }\DataTypeTok{N_s=}\DecValTok{50}\NormalTok{)}
\NormalTok{ag88 <-}\StringTok{ }\KeywordTok{generate_response_smooth}\NormalTok{(}\DataTypeTok{J=}\DecValTok{50}\NormalTok{, }\DataTypeTok{mod=}\StringTok{"glm"}\NormalTok{, }\DataTypeTok{e_sigma=}\DecValTok{8}\NormalTok{, }\DataTypeTok{N_s=}\DecValTok{100}\NormalTok{)}

\NormalTok{al2 <-}\StringTok{ }\KeywordTok{generate_response_smooth}\NormalTok{(}\DataTypeTok{J=}\DecValTok{50}\NormalTok{, }\DataTypeTok{mod=}\StringTok{"lm"}\NormalTok{, }\DataTypeTok{e_sigma=}\DecValTok{2}\NormalTok{, }\DataTypeTok{N_s=}\DecValTok{50}\NormalTok{)}
\NormalTok{al22 <-}\StringTok{ }\KeywordTok{generate_response_smooth}\NormalTok{(}\DataTypeTok{J=}\DecValTok{50}\NormalTok{, }\DataTypeTok{mod=}\StringTok{"lm"}\NormalTok{, }\DataTypeTok{e_sigma=}\DecValTok{2}\NormalTok{, }\DataTypeTok{N_s=}\DecValTok{100}\NormalTok{)}
\NormalTok{al4 <-}\StringTok{ }\KeywordTok{generate_response_smooth}\NormalTok{(}\DataTypeTok{J=}\DecValTok{50}\NormalTok{, }\DataTypeTok{mod=}\StringTok{"lm"}\NormalTok{, }\DataTypeTok{e_sigma=}\DecValTok{4}\NormalTok{, }\DataTypeTok{N_s=}\DecValTok{50}\NormalTok{)}
\NormalTok{al44 <-}\StringTok{ }\KeywordTok{generate_response_smooth}\NormalTok{(}\DataTypeTok{J=}\DecValTok{50}\NormalTok{, }\DataTypeTok{mod=}\StringTok{"lm"}\NormalTok{, }\DataTypeTok{e_sigma=}\DecValTok{4}\NormalTok{, }\DataTypeTok{N_s=}\DecValTok{100}\NormalTok{)}
\NormalTok{al8 <-}\StringTok{ }\KeywordTok{generate_response_smooth}\NormalTok{(}\DataTypeTok{J=}\DecValTok{50}\NormalTok{, }\DataTypeTok{mod=}\StringTok{"lm"}\NormalTok{, }\DataTypeTok{e_sigma=}\DecValTok{8}\NormalTok{, }\DataTypeTok{N_s=}\DecValTok{50}\NormalTok{)}
\NormalTok{al88 <-}\StringTok{ }\KeywordTok{generate_response_smooth}\NormalTok{(}\DataTypeTok{J=}\DecValTok{50}\NormalTok{, }\DataTypeTok{mod=}\StringTok{"lm"}\NormalTok{, }\DataTypeTok{e_sigma=}\DecValTok{8}\NormalTok{, }\DataTypeTok{N_s=}\DecValTok{100}\NormalTok{)}


\NormalTok{bg2 <-}\StringTok{ }\KeywordTok{generate_response}\NormalTok{(}\DataTypeTok{J=}\DecValTok{50}\NormalTok{, }\DataTypeTok{mod=}\StringTok{"glm"}\NormalTok{, }\DataTypeTok{e_sigma=}\DecValTok{2}\NormalTok{, }\DataTypeTok{N_s=}\DecValTok{50}\NormalTok{)}
\NormalTok{bg22 <-}\StringTok{ }\KeywordTok{generate_response}\NormalTok{(}\DataTypeTok{J=}\DecValTok{50}\NormalTok{, }\DataTypeTok{mod=}\StringTok{"glm"}\NormalTok{, }\DataTypeTok{e_sigma=}\DecValTok{2}\NormalTok{, }\DataTypeTok{N_s=}\DecValTok{100}\NormalTok{)}
\NormalTok{bg4 <-}\StringTok{ }\KeywordTok{generate_response}\NormalTok{(}\DataTypeTok{J=}\DecValTok{50}\NormalTok{, }\DataTypeTok{mod=}\StringTok{"glm"}\NormalTok{, }\DataTypeTok{e_sigma=}\DecValTok{4}\NormalTok{, }\DataTypeTok{N_s=}\DecValTok{50}\NormalTok{)}
\NormalTok{bg44 <-}\StringTok{ }\KeywordTok{generate_response}\NormalTok{(}\DataTypeTok{J=}\DecValTok{50}\NormalTok{, }\DataTypeTok{mod=}\StringTok{"glm"}\NormalTok{, }\DataTypeTok{e_sigma=}\DecValTok{4}\NormalTok{, }\DataTypeTok{N_s=}\DecValTok{100}\NormalTok{)}
\NormalTok{bg8 <-}\StringTok{ }\KeywordTok{generate_response}\NormalTok{(}\DataTypeTok{J=}\DecValTok{50}\NormalTok{, }\DataTypeTok{mod=}\StringTok{"glm"}\NormalTok{, }\DataTypeTok{e_sigma=}\DecValTok{8}\NormalTok{, }\DataTypeTok{N_s=}\DecValTok{50}\NormalTok{)}
\NormalTok{bg88 <-}\StringTok{ }\KeywordTok{generate_response}\NormalTok{(}\DataTypeTok{J=}\DecValTok{50}\NormalTok{, }\DataTypeTok{mod=}\StringTok{"glm"}\NormalTok{, }\DataTypeTok{e_sigma=}\DecValTok{8}\NormalTok{, }\DataTypeTok{N_s=}\DecValTok{100}\NormalTok{)}

\NormalTok{bl2 <-}\StringTok{ }\KeywordTok{generate_response}\NormalTok{(}\DataTypeTok{J=}\DecValTok{50}\NormalTok{, }\DataTypeTok{mod=}\StringTok{"lm"}\NormalTok{, }\DataTypeTok{e_sigma=}\DecValTok{2}\NormalTok{, }\DataTypeTok{N_s=}\DecValTok{50}\NormalTok{)}
\NormalTok{bl22 <-}\StringTok{ }\KeywordTok{generate_response}\NormalTok{(}\DataTypeTok{J=}\DecValTok{50}\NormalTok{, }\DataTypeTok{mod=}\StringTok{"lm"}\NormalTok{, }\DataTypeTok{e_sigma=}\DecValTok{2}\NormalTok{, }\DataTypeTok{N_s=}\DecValTok{100}\NormalTok{)}
\NormalTok{bl4 <-}\StringTok{ }\KeywordTok{generate_response}\NormalTok{(}\DataTypeTok{J=}\DecValTok{50}\NormalTok{, }\DataTypeTok{mod=}\StringTok{"lm"}\NormalTok{, }\DataTypeTok{e_sigma=}\DecValTok{4}\NormalTok{, }\DataTypeTok{N_s=}\DecValTok{50}\NormalTok{)}
\NormalTok{bl44 <-}\StringTok{ }\KeywordTok{generate_response}\NormalTok{(}\DataTypeTok{J=}\DecValTok{50}\NormalTok{, }\DataTypeTok{mod=}\StringTok{"lm"}\NormalTok{, }\DataTypeTok{e_sigma=}\DecValTok{4}\NormalTok{, }\DataTypeTok{N_s=}\DecValTok{100}\NormalTok{)}
\NormalTok{bl8 <-}\StringTok{ }\KeywordTok{generate_response}\NormalTok{(}\DataTypeTok{J=}\DecValTok{50}\NormalTok{, }\DataTypeTok{mod=}\StringTok{"lm"}\NormalTok{, }\DataTypeTok{e_sigma=}\DecValTok{8}\NormalTok{, }\DataTypeTok{N_s=}\DecValTok{50}\NormalTok{)}
\NormalTok{bl88 <-}\StringTok{ }\KeywordTok{generate_response}\NormalTok{(}\DataTypeTok{J=}\DecValTok{50}\NormalTok{, }\DataTypeTok{mod=}\StringTok{"lm"}\NormalTok{, }\DataTypeTok{e_sigma=}\DecValTok{8}\NormalTok{, }\DataTypeTok{N_s=}\DecValTok{100}\NormalTok{)}
\end{Highlighting}
\end{Shaded}

\begin{Shaded}
\begin{Highlighting}[]
\CommentTok{### Example 1}

\CommentTok{## GLM_2_50}
\NormalTok{ag2 <-}\StringTok{ }\KeywordTok{generate_response_smooth}\NormalTok{(}\DataTypeTok{J=}\DecValTok{50}\NormalTok{, }\DataTypeTok{mod=}\StringTok{"glm"}\NormalTok{, }\DataTypeTok{e_sigma=}\DecValTok{2}\NormalTok{, }\DataTypeTok{N_s=}\DecValTok{50}\NormalTok{)}
\CommentTok{# generation}
\NormalTok{beta_hat <-}\StringTok{ }\OtherTok{NULL}
\ControlFlowTok{for}\NormalTok{(i }\ControlFlowTok{in} \DecValTok{1}\OperatorTok{:}\DecValTok{50}\NormalTok{)\{}
\NormalTok{  results <-}\StringTok{ }\KeywordTok{granular}\NormalTok{(}\KeywordTok{unlist}\NormalTok{(ag2}\OperatorTok{$}\NormalTok{x_list[i]), }\KeywordTok{unlist}\NormalTok{(ag2}\OperatorTok{$}\NormalTok{y[i]), }\DataTypeTok{mod =} \StringTok{"glm"}\NormalTok{)}
\NormalTok{  beta_hat <-}\StringTok{ }\KeywordTok{rbind}\NormalTok{(beta_hat,results)}
\NormalTok{\}}

\NormalTok{K <-}\StringTok{ }\KeywordTok{make_K}\NormalTok{(ag2}\OperatorTok{$}\NormalTok{z)}

\CommentTok{# multilevel}

\NormalTok{GCV_vec <-}\StringTok{ }\OtherTok{NULL}
\NormalTok{lambda <-}\StringTok{ }\KeywordTok{seq}\NormalTok{(}\FloatTok{0.00005}\NormalTok{, }\FloatTok{0.00012}\NormalTok{, }\DataTypeTok{by =} \FloatTok{1e-6}\NormalTok{)}
\ControlFlowTok{for}\NormalTok{(i }\ControlFlowTok{in} \DecValTok{1}\OperatorTok{:}\KeywordTok{length}\NormalTok{(lambda))\{}
\NormalTok{  EM_out <-}\StringTok{ }\KeywordTok{main_EM}\NormalTok{(}\DataTypeTok{beta_hat_vec =} \KeywordTok{unlist}\NormalTok{(beta_hat[,}\DecValTok{1}\NormalTok{]), }\DataTypeTok{V =} \KeywordTok{diag}\NormalTok{(}\KeywordTok{unlist}\NormalTok{(beta_hat[,}\DecValTok{2}\NormalTok{])), }\DataTypeTok{K =}\NormalTok{ K, lambda[i])}
\NormalTok{  GCV_vec <-}\StringTok{ }\KeywordTok{rbind}\NormalTok{(GCV_vec,EM_out}\OperatorTok{$}\NormalTok{GCV)}
\NormalTok{\}}

\KeywordTok{plot}\NormalTok{(lambda, GCV_vec)}
\end{Highlighting}
\end{Shaded}

\includegraphics{mlspline_code_files/figure-latex/unnamed-chunk-8-1.pdf}

\begin{Shaded}
\begin{Highlighting}[]
\NormalTok{EM_out <-}\StringTok{ }\KeywordTok{main_EM}\NormalTok{(}\DataTypeTok{beta_hat_vec =} \KeywordTok{unlist}\NormalTok{(beta_hat[,}\DecValTok{1}\NormalTok{]), }\DataTypeTok{V =} \KeywordTok{diag}\NormalTok{(}\KeywordTok{unlist}\NormalTok{(beta_hat[,}\DecValTok{2}\NormalTok{])), }\DataTypeTok{K =}\NormalTok{ K, }\DataTypeTok{lambda =} \FloatTok{0.0001}\NormalTok{)}

\NormalTok{RMSE_Multilevel <-}\StringTok{ }\KeywordTok{sqrt}\NormalTok{((}\DecValTok{1}\OperatorTok{/}\DecValTok{50}\NormalTok{)}\OperatorTok{*}\KeywordTok{t}\NormalTok{(ag2}\OperatorTok{$}\NormalTok{true_mu}\OperatorTok{-}\NormalTok{EM_out}\OperatorTok{$}\NormalTok{mu)}\OperatorTok\NormalTok{(ag2}\OperatorTok{$}\NormalTok{true_mu}\OperatorTok{-}\NormalTok{EM_out}\OperatorTok{$}\NormalTok{mu))}

\CommentTok{# naive}

\NormalTok{GCV_vec <-}\StringTok{ }\OtherTok{NULL}
\NormalTok{lambda <-}\StringTok{ }\KeywordTok{seq}\NormalTok{(}\FloatTok{0.00016}\NormalTok{, }\FloatTok{0.00017}\NormalTok{, }\DataTypeTok{by =} \FloatTok{1e-7}\NormalTok{)}
\ControlFlowTok{for}\NormalTok{(i }\ControlFlowTok{in} \DecValTok{1}\OperatorTok{:}\KeywordTok{length}\NormalTok{(lambda))\{}
\NormalTok{  naive_out <-}\StringTok{ }\KeywordTok{naive_ss}\NormalTok{(}\DataTypeTok{beta_hat_vec =} \KeywordTok{unlist}\NormalTok{(beta_hat[,}\DecValTok{1}\NormalTok{]), }\DataTypeTok{lambda =}\NormalTok{ lambda[i], }\DataTypeTok{K =}\NormalTok{ K)}
\NormalTok{  GCV_vec <-}\StringTok{ }\KeywordTok{rbind}\NormalTok{(GCV_vec,naive_out}\OperatorTok{$}\NormalTok{GCV)}
\NormalTok{\}}

\KeywordTok{plot}\NormalTok{(lambda, GCV_vec)}
\end{Highlighting}
\end{Shaded}

\includegraphics{mlspline_code_files/figure-latex/unnamed-chunk-8-2.pdf}

\begin{Shaded}
\begin{Highlighting}[]
\NormalTok{naive_out <-}\StringTok{ }\KeywordTok{naive_ss}\NormalTok{(}\DataTypeTok{beta_hat_vec =} \KeywordTok{unlist}\NormalTok{(beta_hat[,}\DecValTok{1}\NormalTok{]), }\DataTypeTok{lambda =} \FloatTok{0.000168}\NormalTok{, }\DataTypeTok{K =}\NormalTok{ K)}

\NormalTok{RMSE_Naive <-}\StringTok{ }\KeywordTok{sqrt}\NormalTok{((}\DecValTok{1}\OperatorTok{/}\DecValTok{50}\NormalTok{)}\OperatorTok{*}\KeywordTok{t}\NormalTok{(ag2}\OperatorTok{$}\NormalTok{true_mu}\OperatorTok{-}\NormalTok{naive_out}\OperatorTok{$}\NormalTok{mu)}\OperatorTok\NormalTok{(ag2}\OperatorTok{$}\NormalTok{true_mu}\OperatorTok{-}\NormalTok{naive_out}\OperatorTok{$}\NormalTok{mu))}
\end{Highlighting}
\end{Shaded}

\begin{Shaded}
\begin{Highlighting}[]
\CommentTok{### Example 2}

\CommentTok{## GLM_2_50}
\NormalTok{bg2 <-}\StringTok{ }\KeywordTok{generate_response}\NormalTok{(}\DataTypeTok{J=}\DecValTok{50}\NormalTok{, }\DataTypeTok{mod=}\StringTok{"glm"}\NormalTok{, }\DataTypeTok{e_sigma=}\DecValTok{2}\NormalTok{, }\DataTypeTok{N_s=}\DecValTok{50}\NormalTok{)}

\CommentTok{# generation}

\NormalTok{beta_hat <-}\StringTok{ }\OtherTok{NULL}
\ControlFlowTok{for}\NormalTok{(i }\ControlFlowTok{in} \DecValTok{1}\OperatorTok{:}\DecValTok{50}\NormalTok{)\{}
\NormalTok{  results <-}\StringTok{ }\KeywordTok{granular}\NormalTok{(}\KeywordTok{unlist}\NormalTok{(bg2}\OperatorTok{$}\NormalTok{x_list[i]), }\KeywordTok{unlist}\NormalTok{(bg2}\OperatorTok{$}\NormalTok{y[i]), }\DataTypeTok{mod =} \StringTok{"glm"}\NormalTok{)}
\NormalTok{  beta_hat <-}\StringTok{ }\KeywordTok{rbind}\NormalTok{(beta_hat,results)}
\NormalTok{\}}
\end{Highlighting}
\end{Shaded}

\begin{verbatim}
## Warning: glm.fit: fitted probabilities numerically 0 or 1 occurred

## Warning: glm.fit: fitted probabilities numerically 0 or 1 occurred

## Warning: glm.fit: fitted probabilities numerically 0 or 1 occurred

## Warning: glm.fit: fitted probabilities numerically 0 or 1 occurred

## Warning: glm.fit: fitted probabilities numerically 0 or 1 occurred
\end{verbatim}

\begin{Shaded}
\begin{Highlighting}[]
\NormalTok{K <-}\StringTok{ }\KeywordTok{make_K}\NormalTok{(bg2}\OperatorTok{$}\NormalTok{z)}

\CommentTok{# multilevel}

\NormalTok{GCV_vec <-}\StringTok{ }\OtherTok{NULL}
\NormalTok{lambda <-}\StringTok{ }\KeywordTok{seq}\NormalTok{(}\FloatTok{0.000195}\NormalTok{, }\FloatTok{0.0002}\NormalTok{, }\DataTypeTok{by =} \FloatTok{1e-7}\NormalTok{)}
\ControlFlowTok{for}\NormalTok{(i }\ControlFlowTok{in} \DecValTok{1}\OperatorTok{:}\KeywordTok{length}\NormalTok{(lambda))\{}
\NormalTok{  EM_out <-}\StringTok{ }\KeywordTok{main_EM}\NormalTok{(}\DataTypeTok{beta_hat_vec =} \KeywordTok{unlist}\NormalTok{(beta_hat[,}\DecValTok{1}\NormalTok{]), }\DataTypeTok{V =} \KeywordTok{diag}\NormalTok{(}\KeywordTok{unlist}\NormalTok{(beta_hat[,}\DecValTok{2}\NormalTok{])), }\DataTypeTok{K =}\NormalTok{ K, lambda[i])}
\NormalTok{  GCV_vec <-}\StringTok{ }\KeywordTok{rbind}\NormalTok{(GCV_vec,EM_out}\OperatorTok{$}\NormalTok{GCV)}
\NormalTok{\}}

\KeywordTok{plot}\NormalTok{(lambda, GCV_vec)}
\end{Highlighting}
\end{Shaded}

\includegraphics{mlspline_code_files/figure-latex/unnamed-chunk-9-1.pdf}

\begin{Shaded}
\begin{Highlighting}[]
\NormalTok{EM_out <-}\StringTok{ }\KeywordTok{main_EM}\NormalTok{(}\DataTypeTok{beta_hat_vec =} \KeywordTok{unlist}\NormalTok{(beta_hat[,}\DecValTok{1}\NormalTok{]), }\DataTypeTok{V =} \KeywordTok{diag}\NormalTok{(}\KeywordTok{unlist}\NormalTok{(beta_hat[,}\DecValTok{2}\NormalTok{])), }\DataTypeTok{K =}\NormalTok{ K, }\DataTypeTok{lambda =} \FloatTok{0.000197}\NormalTok{)}

\NormalTok{RMSE_Multilevel <-}\StringTok{ }\KeywordTok{sqrt}\NormalTok{((}\DecValTok{1}\OperatorTok{/}\DecValTok{50}\NormalTok{)}\OperatorTok{*}\KeywordTok{t}\NormalTok{(bg2}\OperatorTok{$}\NormalTok{true_mu}\OperatorTok{-}\NormalTok{EM_out}\OperatorTok{$}\NormalTok{mu)}\OperatorTok\NormalTok{(bg2}\OperatorTok{$}\NormalTok{true_mu}\OperatorTok{-}\NormalTok{EM_out}\OperatorTok{$}\NormalTok{mu))}

\CommentTok{# naive}

\NormalTok{GCV_vec <-}\StringTok{ }\OtherTok{NULL}
\NormalTok{lambda <-}\StringTok{ }\KeywordTok{seq}\NormalTok{(}\FloatTok{15.37}\NormalTok{, }\FloatTok{15.375}\NormalTok{, }\DataTypeTok{by =} \FloatTok{0.0001}\NormalTok{)}
\ControlFlowTok{for}\NormalTok{(i }\ControlFlowTok{in} \DecValTok{1}\OperatorTok{:}\KeywordTok{length}\NormalTok{(lambda))\{}
\NormalTok{  naive_out <-}\StringTok{ }\KeywordTok{naive_ss}\NormalTok{(}\DataTypeTok{beta_hat_vec =} \KeywordTok{unlist}\NormalTok{(beta_hat[,}\DecValTok{1}\NormalTok{]), }\DataTypeTok{lambda =}\NormalTok{ lambda[i], }\DataTypeTok{K =}\NormalTok{ K)}
\NormalTok{  GCV_vec <-}\StringTok{ }\KeywordTok{rbind}\NormalTok{(GCV_vec,naive_out}\OperatorTok{$}\NormalTok{GCV)}
\NormalTok{\}}

\KeywordTok{plot}\NormalTok{(lambda, GCV_vec)}
\end{Highlighting}
\end{Shaded}

\includegraphics{mlspline_code_files/figure-latex/unnamed-chunk-9-2.pdf}

\begin{Shaded}
\begin{Highlighting}[]
\NormalTok{naive_out <-}\StringTok{ }\KeywordTok{naive_ss}\NormalTok{(}\DataTypeTok{beta_hat_vec =} \KeywordTok{unlist}\NormalTok{(beta_hat[,}\DecValTok{1}\NormalTok{]), }\DataTypeTok{lambda =} \FloatTok{15.374}\NormalTok{, }\DataTypeTok{K =}\NormalTok{ K)}

\NormalTok{RMSE_Naive <-}\StringTok{ }\KeywordTok{sqrt}\NormalTok{((}\DecValTok{1}\OperatorTok{/}\DecValTok{50}\NormalTok{)}\OperatorTok{*}\KeywordTok{t}\NormalTok{(bg2}\OperatorTok{$}\NormalTok{true_mu}\OperatorTok{-}\NormalTok{naive_out}\OperatorTok{$}\NormalTok{mu)}\OperatorTok\NormalTok{(bg2}\OperatorTok{$}\NormalTok{true_mu}\OperatorTok{-}\NormalTok{naive_out}\OperatorTok{$}\NormalTok{mu))}
\end{Highlighting}
\end{Shaded}

\hypertarget{problem}{%
\subsection{problem}\label{problem}}

Setting : Model GLM, \(\tau=2, \; N_{min}=50\)

We can get \(RMSE\)s multilevel approach suffers under GLM.
\(RMSE_1=3.21.(\lambda=0.00023),\; RMSE_2=3.05(\lambda=0.000395), RMSE_3=3.43(\lambda=0.00062), \;RMSE_4=2.98(\lambda=0.000508),\;RMSE_5=3.04(\lambda=0.000197)\)

We can get \(RMSE\)s naive approach suffers under GLM.
\(RMSE_1=12.9(\lambda=46.4),\;RMSE_2=16.3(\lambda=54.3),\;RMSE_3=76.8(\lambda=93.465),\;RMSE_4=7.2(\lambda=11.2)\;,RMSE_5=22.6(\lambda=394.54)\)

\end{document}
